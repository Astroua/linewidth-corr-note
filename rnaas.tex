%% from aastex61.cls with a few tweaks to allow for the unique format required.
\documentclass{rnaastex}

\begin{document}

\title{Radio line broadening from a spectral response function}

%% Note that the corresponding author command and emails has to come
%% before everything else. Also place all the emails in the \email
%% command instead of using multiple \email calls.
\correspondingauthor{Eric Koch}
\email{ekoch@ualberta.ca}
\author[0000-0001-9605-780X]{Eric Koch}
\author[0000-0002-5204-2259]{Erik Rosolowsky}
\affiliation{University of Alberta, Department of Physics 4-183 CCIS, Edmonton AB T6G 2E1, Canada}

\author{Adam Leroy}
\affiliation{The Ohio State University, Department of Astronomy, 140 West 18th Avenue, Columbus, OH 43210, USA}


%% Note that RNAAS manuscripts DO NOT have abstracts.
%% See the online documentation for the full list of available subject
%% keywords and the rules for their use.
\keywords{}

%% Start the main body of the article. If no sections in the
%% research note leave the \section call blank to make the title.

\section{}

% - Intro: define the domain (radio), the technology (correlation spectrometers), and the problem (<~2 channels across a line)

Spectral-line observations are systematically broadened by the response function of the spectrometer. Although analyses in the radio and submillimetre often assume independent top-hat channels, the spectral response function is a combination of instrumentation and signal processing, just like at other wavelengths.  Typical observing strategies resolve lines with $\sim3\mbox{--}5$ per full-width-half-max, which will bias spectral-line properties---including line width measurements--- without accounting for the spectral response.  In this note, we compare different methods for inferring the line width of a single noiseless Gaussian and how the spectral response affects them.

% We explore how line width estimates are affected without accounting for the spectral response and show that correction factors from the literature do not remove this bias.

We consider a noiseless single Gaussian with width $\sigma$ as the simplest model. Using an ideal spectrometer, each channel is independent and the Gaussian is smoothed with a top-hat kernel one channel in width ($\Delta v$). This smoothing gives the observed spectrum a lower amplitude and larger $\sigma$, which becomes more extreme as $\Delta v \rightarrow \sigma$.

% Spectral-lines measured by an ideal spectrometer\footnote{Each channel is independent of each other.} are a weighted average of the line profile over the finite spectral channel.

% Though independent, top-hat spectral channels are commonly assumed,
Real radio observations have a more complex spectral response than a top-hat.
Signals measured by a spectrometer are truncated by a top-hat function set by the maximum time lag, the inverse of which set the spectral resolution.  This truncation gives a sinc response in frequency, and signal not aligned to the channel centres will exhibit ``ringing''. Apodizing kernels, such as the Hann kernel, reduce this ringing at the expense of correlating nearby channels\footnote{\url{https://safe.nrao.edu/wiki/pub/Main/ALMAWindowFunctions/Note_on_Spectral_Response.pdf}}.  Some instruments have well-characterized response functions that can be modelled for in observed spectra \citep{rosolowsky2008}.  Without this information, a three-element Hann-like kernel ($[k, 1 - 2k, k]$ with channel-coupling $k$) can approximate the nearest-neighbour channel correlations \citep{leroy2016}.

The top row in Figure \ref{fig:width_recovery_comparison} show a Gaussian averaged over independent channels and convolved with a Hann-like kernel with $\Delta v=\sigma$.  The two panels exhibit how line broadening changes when the line is positioned in the centre and edge of a channel. The former minimally broadens the Gaussian, while the latter is the worst-case of broadening.  The peak location depends on the source and cannot be assumed beforehand, thus requiring the spectral response to be accounted for to recover the correct line properties.

% For some instruments, the response function is well-characterized \citep{rosolowsky2008}, particularly for instruments whose response functions significantly alters the observed line shape \citep[e.g.][]{martin2015}.

% The line broadening can be forward-modelled by channel-averaging and convolving the fit model with the spectral response function prior to comparing with the observed spectrum. Forward-modelling of spectrum lines has been used for some radio observations \citep{rosolowsky2008} and is commonly used for shorter wavelength observations \citep[e.g.,][]{martin2015}.

% - How to measure line width: fits, fits with forward model, equivalent width, second moment.

To recover the line properties, the spectral response can be included in the model for the observed spectrum.  We forward-model the spectral response by averaging the Gaussian over the channels and convolving with a Hann-like kernel ($k=0.11$) prior to fitting the observed spectrum.  Forward-modelling the spectral response is common-practice in many fields \citep[e.g.,][]{martin2015}, but is seldom used in radio studies.  Because no noise is added, this model is defined to recover the correct line properties.

While forward-modelling accounts for the spectral response, the spectrum must be fit, which requires knowing a correct model and may be affected by noise.  Instead, many studies adopt approximate methods for the line width assuming a Gaussian shape. We compare the forward-model fit with four methods:
\begin{enumerate}
    \item {\bf Fit to Gaussian} without forward-modelling.
    \item {\bf Moment 2}: $\sigma_{\rm Mom2}=\sqrt{\sum_i T_i (v_i - v_0)^2 / \sum_i T_i}$ where $T_i$ is the value at $v_i$ and $v_0$ is the centroid.
    \item {\bf Equivalent width}: $\sigma_{\rm equiv} = T_{\rm peak} / \sqrt{2\pi} \Sigma$ where $T_{\rm peak}$ is the measured amplitude and $\Sigma= \sum_i T_i \Delta v$ \citep{heyer2001,leroy2016,sun2018}.
    \item {\bf Half-width-half-max (HWHM)} estimated from where the observed spectrum is half the peak \citep{stilp2013a,stilp2013b,koch2018}.
\end{enumerate}

% When approximate line width methods or fitting without forward-modelling are used,
Many radio studies correct the measured line width based on a top-hat kernel, assumed to equal the area of a Gaussian equal to the rectangular area over one channel. This factor is subtracted in quadrature from the measured line width \citep{cprops}.  To account for channel correlations, \citet{leroy2016} extended this correction factor with an empirical calibration relating channel-to-channel correlations ($r$) to a Hann-like kernel:
% ($k\approx 0.0 + 0.47r - 0.23r^2 - 0.16r^3 + 0.43 r^4$).
% This empirical calibration gives a line broadening correction of:
\begin{equation}
    \label{eq:leroy16_corrfact}
    \sigma_{\rm chan} \approx \frac{\Delta v}{\sqrt{2\pi}} \left( 1.0 + 1.18k + 10.4 k^2 \right).
\end{equation}
As $k\rightarrow0$, the term in brackets approaches unity to the assumed correction factor of a top-hat response.  We adopt $k=0.11$ for our example\citep[see][]{sun2018}.

% \citet{sun2018} give the channel correlations for a number of CO datasets of nearby galaxies; we adopt $k=0.11$ for our example.

% For the example here, we use $r=0.26$ ($k=0.11$) for the IRAM-30m CO(2-1) data of M33 \citep{druard2014}.
% Using this approximation for a response function, we convolve the channel-averaged Gaussian (Eq. \ref{eq:finite_gaussian}) in Figure \ref{fig:width_recovery_comparison} to show how a realistic response function further broadens the Gaussian.

We generate model spectra varying the sampling $\Delta v / \sigma$ and calculate the line width from each method, with and without subtracting Eq. \ref{eq:leroy16_corrfact}. We also compare observed spectra with independent channels (top-hat) and correlated by a Hann-like kernel. The bottom four panels in Figure \ref{fig:width_recovery_comparison} show the measured line widths.

When the line is centered on a channel (first column), the line width methods are similarly biased and convolving with a Hann-like kernel only broadens the spectra.  Line widths are over-corrected with Eq. \ref{eq:leroy16_corrfact} when $k=0$, but are within $<5\%$ when correlated with the Hann-like kernel for $\sigma/\Delta v \geq 1$.

There are deviations between the methods when the line is centered on the channel edge (second column). With independent and correlated channels, the Gaussian fit and second moment are biased similarly to when the peak is centered on a channel.  The equivalent width and HWHM are biased to larger values than the latter methods because they depend on the amplitude of the profile, which is underestimated more than in the channel-centre case. Eq. \ref{eq:leroy16_corrfact} {\it underestimates} the line broadening in this case. Methods that leverage information across many channels are preferable over those that use only the line core when $\sigma / \Delta v < 2$. 
% This discrepancy between methods suggests that, when a line is marginally resolved ($\sigma / \Delta v < 2$), methods that leverage information across many channels are vastly preferable over those that focus on the line core (equivalent width and HWHM).

% When a Hann-like kernel is applied to lines at channel edge, the line widths are qualitatively the same as those with a top-hat kernel.  Equation \ref{eq:leroy16_corrfact} ($k=0.11$) works well for the Gaussian fit and second moment for $\sigma/\Delta v \geq 1$, but underestimates the line broadening for the equivalent width and HWHM.

% XXX Mention maximum uncertainty from channel edge vs. expected uncertainties from randomly centering the line within a channel XXX.
% The examples shown here demonstrate the best (channel centre) and worst (channel edge) cases of recovering spectral properties.  Assuming that

% When using approximate methods to find the line width, our results show: 
Our results show:
\begin{enumerate}
    \item The correction factor most often used in radio analyses (Eq. \ref{eq:leroy16_corrfact}; $k=0$) does not correct for line width broadening in any case.  $\Delta v / \sqrt{2\pi}$ is not the correct line width contribution for a top-hat kernel.
    \item A spectral response function that is more complex than a top-hat kernel needs to be accounted for.  In our example, a Hann-like kernel increased measured line widths by $\sim10\%$ compared to the top-hat kernel.
    \item To measure line widths within $<5\%$ of the actual line width for correlated channels, channel widths of $\sigma / \Delta v \geq 2$ are needed for fitting a Gaussian and the second moment, while $\sigma / \Delta v \geq 4$ is needed for the equivalent width and HWHM. Broadening from the spectral response can be considered negligible above these limits.  Where possible, we recommend that observational setups sample $\sigma / \Delta v > 2$ (or $5$ channels per FWHM).
    \item When fitting an analytical model to observed data, it is vastly preferable to forward-model with a spectral response function when $\sigma / \Delta v \leq 2$.
    \item Methods that estimate the line width with information over multiple channels (fitting, second moment) are preferable over those that use only the line core (equivalent width, HWHM). These latter methods can place upper limits on the line width, but should be treated with caution when $\sigma / \Delta v <4$.
\end{enumerate}

We stress that these recommendations are only for {\it high signal-to-noise, Gaussian spectra}. Noise or multi-component spectra will change the biases for these methods and can be explored for individual data sets.
% We also note that these methods will break down when the Gaussian assumption is invalid \citep{koch2018}.

Code for this note is available at: \url{https://github.com/Astroua/linewidth-corr-note}.

% In all cases, the only method that correctly accounts for line width broadening is fitting with forward-modelling.  We recommend that this method be used to recover line widths whenever feasible.

% The examples in Figure \ref{fig:width_recovery_comparison} do not include noise in the spectra.  We test the forward-modelling model in the presence of noise drawn to give a peak signal-to-noise of 5 in the spectrum.  The noisy spectrum is then convolved with the response function used above to correlate the noise.  Drawing 1000 iterations of noise, we find that the fitted parameters are not biased by line broadening.  However, the Levenberg-Marquadt algorithm used for the fitting assumes that the noise is uncorrelated.  We test whether breaking this assumption significantly underestimates the parameter uncertainties from the covariance matrix by calculating the fraction of iterations that the true parameter value lies within the $1\mbox{-}\sigma$ uncertainty range from the fit.  We find that true values are within the uncertainty range in $\sim70\%$ of the iterations, similar to the expected $68.2\%$ for the $1\mbox{-}\sigma$ range for a Gaussian distribution.  The lack of modelling for correlated errors should not significantly change the uncertainties.  The correlated noise may become more important if the spectral response function correlates channels beyond their nearest neighbours.  In that case, a Gaussian process can be used to model the correlated uncertainties XXX rasmussen and michaels XXX.


\begin{figure*}
\includegraphics[width=\textwidth]{combined_figure}
\caption{\label{fig:width_recovery_comparison} Top row: A Gaussian model (blue-solid) sampled in channels $\Delta v = \sigma$ (green-dashed) and convolved with a Hann-like kernel (orange dot-dashed) with the peak at the centre (left) and edge of a channel (right). Centre and bottom rows: Line widths measured with different methods as a function of $\Delta v / \sigma$ (solid lines) when the spectrum peaks at the channel centre (left) and edge (right). The middle row are spectra with independent channels and the bottom row is convolved with a Hann-like kernel ($k=0.11$). Dashed lines are line widths corrected with Eq. \ref{eq:leroy16_corrfact}.}
% The first column shows results for a Gaussian at the channel centre and the second column is a Gaussian centred at the channel edge.  Where the peak is located with respect to the channels alters the line broadening.  We find that forward-modelling is the only method that correctly accounts for line broadening in all cases. XXX Change the side labels to independent channels and channels correlated with Hann-like kernel XXX}

\end{figure*}

\acknowledgments

EWK and EWR acknowledge the support of the Natural Sciences and Engineering Research Council of Canada (RGPIN-2017-03987).

\software{astropy \citep{astropy}, matplotlib \citep{mpl}}

\begin{thebibliography}{}

% \bibitem[Druard et al.(2014)]{druard2014} Druard, C., Braine, J., Schuster, K.~F., et al.\ 2014, \aap, 567, A118.

\bibitem[Astropy Collaboration et al.(2018)]{astropy} Astropy Collaboration, Price-Whelan, A.~M., Sip{\'{o}}cz, B.~M., et al.\ 2018, \aj, 156, 123.

\bibitem[Heyer et al.(2001)]{heyer2001} Heyer, M.~H., Carpenter, J.~M., \& Snell, R.~L.\ 2001, \apj, 551, 852.

\bibitem[Hunter(2007)]{mpl} Hunter, J.~D.\ 2007, Computing in Science and Engineering, 9, 90.

\bibitem[Koch et al.(2018)]{koch2018} Koch, E.~W., Rosolowsky, E.~W., Lockman, F.~J., et al.\ 2018, \mnras, 479, 2505.

\bibitem[Leroy et al.(2016)]{leroy2016} Leroy, A.~K., Hughes, A., Schruba, A., et al.\ 2016, \apj, 831, 16.

\bibitem[Martin et al.(2015)]{martin2015} Martin, T., Drissen, L., \& Joncas, G.\ 2015, Astronomical Data Analysis Software an Systems XXIV (ADASS XXIV), 327.

\bibitem[Rosolowsky, \& Leroy(2006)]{cprops} Rosolowsky, E., \& Leroy, A.\ 2006, Publications of the Astronomical Society of the Pacific, 118, 590.

\bibitem[Rosolowsky et al.(2008)]{rosolowsky2008} Rosolowsky, E.~W., Pineda, J.~E., Foster, J.~B., et al.\ 2008, The Astrophysical Journal Supplement Series, 175, 509.

\bibitem[Stilp et al.(2013a)]{stilp2013a} Stilp, A.~M., Dalcanton, J.~J., Warren, S.~R., et al.\ 2013, \apj, 765, 136.

\bibitem[Stilp et al.(2013b)]{stilp2013b} Stilp, A.~M., Dalcanton, J.~J., Skillman, E., et al.\ 2013, \apj, 773, 88.

\bibitem[Sun et al.(2018)]{sun2018} Sun, J., Leroy, A.~K., Schruba, A., et al.\ 2018, \apj, 860, 172.


\end{thebibliography}

\end{document}
